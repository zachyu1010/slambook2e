% 这章最好是“第零章”或者在序言里
\chapter{预备知识}
\section{本书讲什么}

这是一本介绍视觉SLAM的书,也很可能是第一本以视觉SLAM为主题的中文书。

那么,SLAM是什么?

SLAM是\textbf{S}imultaneous \textbf{L}ocalization \textbf{a}nd \textbf{M}apping的缩写,中文译作“\textbf{同时定位与地图构建}”\textsuperscript{\cite{Liu2016}}。它是指搭载特定\textbf{传感器}的主体,在\textbf{没有环境先验信息}的情况下,于\textbf{运动过程中}建立\textbf{环境}的模型,同时估计自己的\textbf{运动}\textsuperscript{\cite{Davison2007}}。如果这里的传感器主要为相机,那就称为“视觉SLAM”。

本书的主题就是视觉SLAM。这里我们刻意把许多个定义放到一句话中,希望帮助读者建立一个较明确的概念。首先,SLAM的目的是解决“定位”与“地图构建”这两个问题。也就是说,一边要估计传感器自身的位置,一边要建立周围环境的模型。那么怎么解决呢?这需要用到传感器的信息。传感器以一定形式观察外部世界,但不同传感器观察的方式是不同的。之所以要花一本书的内容去讨论这个问题,是因为它很难——特别是我们希望\textbf{实时地}、在\textbf{没有先验知识}的情况下进行SLAM。当用相机作为传感器时,我们要做的就是根据一张张连续运动的图像(它们形成了一段视频),从中推断相机的运动,以及周围环境的情况。

这似乎是个很直观的问题。我们自己走进陌生的环境时不就是这么做的吗?

在计算机视觉(Computer Vision)创立之初,人们就想象着有朝一日,计算机将和人一样,通过眼睛去观察世界,理解周遭的物体,探索未知的领域——这是一个美妙而又浪漫的梦想,吸引了无数的科研人员日夜为之奋斗\textsuperscript{\cite{Hartley2003}}。我们曾经以为这件事情并不困难,然而进展却远不如预想的那么顺利。我们眼中的花草树木、虫鱼鸟兽,在计算机中却是那样的不同:它们只是一个个由数字排列而成的矩阵。让计算机理解图像的内容,就像让我们自己理解这些数字一样困难。我们既不了解自己如何理解图像,也不知道计算机该如何理解、探索这个世界。于是我们困惑了很久,直到几十年后的今天,才发现了一点点成功的迹象:通过人工智能(Artificial Intelligence)中的机器学习(Machine Learning)技术,计算机渐渐能够辨别出物体、人脸、声音、文字——尽管它所用的方式(统计建模)与我们是如此不同。另一方面,在SLAM发展了将近30年之后,我们的相机才渐渐开始能够认识到自身的位置,发觉自己在运动——虽然方式还是和人类有巨大的差异。不过,至少研究者们已经成功地搭建出种种实时SLAM系统,有的能够快速跟踪自身位置,有的甚至能够进行实时的三维重建。

这件事情确实很困难,但我们已经有了很大的进展。更令人兴奋的是,近年来随着科技的发展,涌现出了一大批与SLAM相关的应用点。在许多地方,我们都希望知道自身的位置:室内的扫地机和移动机器人需要定位,野外的自动驾驶汽车需要定位,空中的无人机需要定位,虚拟现实和增强现实的设备也需要定位。SLAM是那样重要。没有它,扫地机就无法在房间自主地移动,只能盲目地游荡;家用机器人就无法按照指令准确到达某个房间;虚拟现实也将永远固定在座椅之上——所有这些新奇的事物都无法出现在现实生活中,那将多么令人遗憾。

今天的研究者和应用开发人员,逐渐意识到了SLAM技术的重要性。在国际上,SLAM已经有近三十年的研究历史,也一直是机器人和计算机视觉的研究热点。21世纪以来,以视觉传感器为中心的\textbf{视觉SLAM技术},在理论和实践上都经历了明显的转变与突破,正逐步从实验室研究迈向市场应用。同时,我们又遗憾地发现,至少在国内,与SLAM相关的论文、书籍仍然非常匮乏,让许多对SLAM技术感兴趣的初学者无从一窥门径。虽然SLAM的理论框架基本趋于稳定,但其编程实现仍然较为复杂,有着较高的技术门槛。刚步入SLAM领域的研究者,不得不花很长的时间,学习大量的知识,往往要走过许多弯路才得以接近SLAM技术的核心。

本书全面系统地介绍了以视觉传感器为主体的视觉SLAM技术,我们希望它能(部分地)填补这方面资料的空白。我们会详细地介绍SLAM的理论背景、系统架构,以及各个模块的主流做法。同时,\textbf{极其重视实践}:本书介绍的\textbf{所有}重要算法,都将给出可以运行的实际代码,以求加深读者的理解。在第二版中,我们对于大多数算法,会讨论内在的原理,而非简单地从函数库中进行调用。之所以这么做,主要是考虑到SLAM毕竟是一项和实践紧密相关的技术。再漂亮的数学理论,如果不能转化为可以运行的代码,那就仍是可望而不可即的空中楼阁,没有实际意义。我们相信,实践出真知,实践出真爱。只有实际地演算过各种算法之后,你才能真正地认识SLAM,真正地喜欢上科研。

自1986年提出以来\textsuperscript{\cite{Smith1986}},SLAM一直是机器人领域的热点问题。关于它的文献数以千计,想要对SLAM发展史上的所有算法及变种做一个完整的说明,是十分困难而且没有必要的。本书中会介绍SLAM所牵涉的背景知识,例如射影几何、计算机视觉、状态估计理论、李群李代数等,并在这些背景知识之上,给出SLAM这棵大树的主干,而略去一部分形状奇特、纹理复杂的枝叶。我们认为这种做法是有效的。如果读者能够掌握主干的精髓,那么自然会有能力去探索那些边缘的、细节的、错综复杂的前沿知识。所以,我们的目的是,让SLAM的初学者通过阅读本书快速地成长为能够探索这个领域边缘的研究者。另一方面,即便你已经是SLAM领域的研究人员,本书也可能有一些你还觉得陌生的地方,可以让你产生新的见解。

目前,与SLAM相关的书籍主要有《概率机器人》(\textit{Probabilistic robotics})\textsuperscript{\cite{Thrun2005}}、《计算机视觉中的多视图几何》(\textit{Multiple View Geometry in Computer Vision})\textsuperscript{\cite{Hartley2003}}、《机器人学中的状态估计》(\textit{State Estimation for Robotics: A Matrix-Lie-Group Approach})\textsuperscript{\cite{Barfoot2016}}等。它们内容丰富、论述全面、推导严谨,是SLAM研究者中脍炙人口的经典教材。然而就目前来看,还存在两个重要的问题:其一,这些图书的目的在于介绍基础理论,SLAM只是其应用之一。因此,它们并不能算是专门讲解SLAM的书籍。其二,它们的内容偏重于数学理论,基本不涉及编程实现,导致读者经常出现“书能看懂却不会编程”的情况。而我们认为,只有读者亲自实现了算法,调试了各个参数,才能谈得上真正理解了问题本身。

我们会提及SLAM的历史、理论、算法、现状,并把完整的SLAM系统分成几个模块:视觉里程计、后端优化、建图,以及回环检测。我们将陪着读者一点点实现这些模块中的核心部分,探讨它们在什么情况下有效,什么情况下会出问题,并指导大家在自己的机器上运行这些代码。你会接触到一些\textbf{必要的}数学理论和许多编程知识,会用到Eigen、OpenCV、PCL、g2o、Ceres等库\footnote{如果你完全没有听说过它们,那么应该感到兴奋,这说明你会从本书中收获很多知识。},掌握它们在Linux操作系统中的使用方法。

从写作风格上,我们不想把本书写成枯燥的理论书籍。技术类图书应该是严谨可靠的,但严谨不意味着刻板。一本优秀的技术书应该是生动有趣而易于理解的。如果你觉得“这个作者怎么这么不正经”,敬请原谅,因为我并不是一个非常严肃的人\footnote{你会经常在脚注中发现一些神奇的东西。}。无论如何,有一件事是可以肯定的:只要你对这门新技术感兴趣,在学习本书的过程中肯定会有所收获!您会掌握与SLAM相关的理论知识,你的编程能力也将有明显的进步。在很多时候,您会有一种“我在陪你一起做科研”的感觉,这正是我所希望的。但愿您能在此过程中发现研究的乐趣,喜欢这种“通过一番努力,看到事情顺利运行”的成就感。

好了,话不多说,祝你旅行愉快!

\section{如何使用本书}
\subsection{组织方式}
本书名为“视觉SLAM十四讲”。顾名思义,我们会像在学校里讲课那样,以“讲”作为本书的基本单元。每一讲都对应一个固定的主题,其中会穿插“理论部分”和“实践部分”两种内容。通常是理论部分在前,实践部分在后。在理论部分中,我们将介绍\textbf{理解算法所必需}的数学知识,并且大多数时候以叙述的方式,而不是像数学教科书那样用“定义—定理—推论”的方式,因为我们觉得这样的方式阅读起来更容易一些,尽管有时候显得不那么严谨。实践部分主要是编程实现,讨论程序里各部分的含义及实验结果。看到标题中带有“实践”两个字的章节,你就应该(兴致勃勃地)打开电脑,和我们一起愉快地编写代码了。

值得一提的是,我们只会把与解决问题相关的数学知识放在书里,并尽量保持浅显。因为我是工科生,所以要勇敢地承认,某些做法只要经验上够用,没必要非得在数学上追求完备。只要我们知道这些算法在绝大多数实际场景下能够工作,并且数学家们(通过冗长而且复杂的证明和讨论)说明在什么情况下可能不工作,那么我就表示满意,而不刻意追究那些看似完美的证明(当然它们固有自己不可否认的价值)。由于SLAM牵涉到了太多数学背景,为了防止使本书变成数学教科书,我们把一些细节上的推导和证明留作习题和补充阅读材料,方便感兴趣的读者进一步阅读参考文献,更深入地掌握相关细节。

每一讲正文之后,我们设计了一些习题。其中,带$^*$号的习题是具有一定难度的。我们强烈建议读者把习题都练习一遍,这对你掌握这些知识很有帮助\footnote{它们也可能成为今后相关行业的面试题,或许还能帮你在找工作时留个好印象。}。

全书内容主要分为两个部分。
\begin{enumerate}
\item 第一部分为\textbf{数学基础}篇,我们会以浅显易懂的方式,铺垫与视觉SLAM相关的数学知识,包括:

\begin{itemize}[leftmargin=1.5em]
	\item 第1讲是前言,介绍这本书的基本信息,习题部分主要包括一些自测题。
	\item 第2讲为SLAM系统概述,介绍一个SLAM系统由哪些模块组成,各模块的具体工作是什么。实践部分介绍编程环境的搭建过程以及IDE的使用。
	\item 第3讲介绍三维空间运动,你将接触到旋转矩阵、四元数、欧拉角的相关知识,并且在Eigen当中使用它们。
	\item 第4讲为李群和李代数。即便你现在不懂李代数为何物,也没有关系。你将学到李代数的定义和使用方式,然后通过Sophus操作它们。
	\item 第5讲介绍针孔相机模型以及图像在计算机中的表达。你将用OpenCV来调取相机的内外参数。
	\item 第6讲介绍非线性优化,包括状态估计理论基础、最小二乘问题、梯度下降方法。你会完成一个使用Ceres和g2o进行曲线拟合的实验。
\end{itemize}

这些就是我们要用到的所有数学知识了,当然,其中还隐含了你以前学过的高等数学和线性代数。我保证它们看起来都不会很难,至少没有听上去那么难。当然,若你想进一步深入挖掘,我们会提供一些参考资料供你阅读,那些材料可能会比正文里讲的知识难一些。

%\clearpage
\item 第二部分为\textbf{SLAM技术}篇。我们会使用第一部分所介绍的理论,讲述视觉SLAM中各个模块的工作原理。

\begin{itemize}[leftmargin=1.5em]
	\item 第7讲为特征点法的视觉里程计。该讲内容比较多,包括特征点的提取与匹配、对极几何约束的计算、PnP和ICP等。在实践中,你将用这些方法去估计两个图像之间的运动。
	\item 第8讲为直接法的视觉里程计。你将学习光流和直接法的原理,然后实现一个简单的直接法运动估计。
	\item 第9讲为后端优化,主要为对Bundle Adjustment的深入讨论,包括基本的BA,以及如何利用稀疏性加速求解过程。你将用Ceres和g2o分别书写一个BA程序。
	\item 第10讲主要讲后端优化中的位姿图。位姿图是表达关键帧之间约束的一种更紧凑的形式。我们会介绍SE(3)和Sim(3)的位姿图,同时你将使用g2o对一个位姿球进行优化。
	\item 第11讲为回环检测,主要介绍以词袋方法为主的回环检测。你将使用dbow3书写字典训练程序和回环检测程序。
	\item 第12讲为地图构建。我们会讨论如何使用单目进行稠密深度图的估计(以及这是多么不可靠),然后讨论RGB-D的稠密地图构建过程。你会书写极线搜索与块匹配的程序,然后在RGB-D中遇到点云地图和八叉树地图的构建问题。
	\item 第13讲是工程章,你将搭建一个双目视觉里程计框架,综合运用先前学过的知识,实现它的基本功能。这个过程中,你会碰到一些问题,例如优化的必要性、关键帧的选择等。我们会在Kitti数据集上测试它的性能,讨论一些改进的手段。
	\item 第14讲主要介绍当前的开源SLAM项目以及未来的发展方向。相信在阅读了前面的知识之后,你会更容易理解它们的原理,实现自己的新想法。
\end{itemize}
\end{enumerate}

最后,如果你完全看不懂上面在说什么,那么恭喜你!这本书很适合你!加油!

\subsection{代码}
本书所有源代码均托管在GitHub上:

{\hfill\url{https://github.com/gaoxiang12/slambook2}\hfill}

\textbf{注意后面有一个2,表示这是第二版书的代码}。我强烈建议读者下载代码以供随时查看。代码是按章节划分的,比如,第7讲的内容就会放在ch7文件夹中。此外,对于书中用到的一些小型库,会以压缩包的形式放在3rdparty文件夹下。在第二版中,我们用git submodule工具来保证读者使用的软件 版本与书中的完全一致,所以读者不必操心软件版本问题。对于像OpenCV那种大中型库,我们会在它们第一次出现时介绍其安装方法。如果你对代码有任何疑问,请单击GitHub上的Issues按钮,提交问题。如果确实是代码出现问题,我们会及时进行修改;即使是你的理解有偏差,我也会尽可能回复。如果你不习惯使用Git,那么单击右侧包含download字样的按钮将代码下载至本地即可。%网上能搜索到很多关于git和github的使用教程。

\subsection{面向的读者}
本书面向对SLAM感兴趣的学生和研究人员。阅读本书需要一定的基础,我们假设你具备以下知识:

\begin{itemize}
\item {\textbf{高等数学、线性代数、概率论}。} 这些是大部分读者应该在大学本科阶段接触过的基本数学知识\footnote{实际当中每个人都至少需要学三遍线性代数:本科一遍,研究生一遍,工作时期一遍。}。你应当明白矩阵和向量是什么,或者做微分和积分是什么意思。对于SLAM中用到的专业知识,我们会额外加以介绍。

\item {\textbf{C++语言基础}。} 由于我们采用C++作为编码语言,所以建议读者至少熟悉这门语言的语法。比如,你应该知道类是什么,如何使用C++标准库,模板类如何使用,等等。我们会避免过多地使用技巧,但有些地方确实无法避免。此外,我们还使用了一些C++ 11标准的内容,不过,我们会在用到的地方加以解释。

\item {\textbf{Linux基础}。} 我们的开发环境是Linux而非Windows,并且只提供Linux下的源程序,\textbf{不会}再提供Windows下的开发方法介绍。\textbf{我们认为,掌握Linux是一个SLAM研究人员所必需的,请初学者暂时不要问为什么,把本书的知识学好之后相信你会和我们有同样的想法。}各种程序库在Linux下的配置都非常便捷,你也会在此过程中体会到Linux的便利。如果读者此前从未使用过Linux,那么最好找一本Linux的教材稍加学习(掌握基本知识即可,一般就是相关图书的前面几章内容)。我们不要求读者具备多么高超的Linux操作技能,但希望读者至少知道“打开终端,进入代码目录”是如何操作的。本讲的习题里有一些Linux知识自测题,如果你清楚自测题的答案,那么阅读本书代码不会有任何问题。
\end{itemize}

对SLAM感兴趣但不具备上述知识的读者,可能在阅读本书时会感到困难。如果你不了解C++的基本知识,可以读一点\emph{C++ Primer Plus}之类的图书入门;如果你缺少相关的数学知识,也可以先阅读一些相关数学教材补充知识,不过我们认为,对大多数大学本科水平的朋友,读懂本书所需的数学背景肯定是具备了。代码方面,你最好花点时间亲自输入一遍,再调节里面的参数,看看效果会发生怎样的改变。这会对学习很有帮助。

本书可作为SLAM相关课程的教材,亦可作为课外自学材料使用。

\section{风格约定}
本书既有数学理论介绍,也有编程实现,因此,为方便阅读,对不同内容采用了不同排版方式加以区分。
\begin{enumerate}
\item 数学公式单独列出,重要的公式还在右侧标了序号,例如:
\begin{equation}
 \bm{y} =\bm{A}\bm{x}.
\end{equation}
数学字体采用国标风格。标量使用斜体字(如$a,\alpha$),向量和矩阵使用粗斜体(如$\bm{a}, \bm{A},\boldsymbol{\Sigma}$,希腊字母除外)。空心粗体代表特殊集合,如实数集$\mathbb{R}$、整数集$\mathbb{Z}$。李代数部分使用哥特体,如$\mathfrak{so}(3),\mathfrak{se}(3)$。
\item 程序代码以方框框出,使用不同的字体和小一些的字号,左侧带有行号。如果程序较长,方框会延续到下一页。总之,看起来像这样:
\begin{lstlisting}[language=C++,caption=示例代码]
#include <iostream>
using namespace std;

int main ( int argc, char** argv )
{
	cout<<"Hello"<<endl;
	return 0;
}
\end{lstlisting}

\item 当代码数量较多或有的部分与之前列出的重复,不适合完全列在书中时,我们会\textbf{仅给出重要片段},并以“片段”二字注明。因此,再说一遍,我们强烈建议读者到GitHub上下载所有源代码,完成练习,以更好地掌握本书知识。

\item 由于排版原因,书中展示的代码可能与GitHub中的代码有稍许不同,请以GitHub上的代码为准。

\item 我们用到的每个库,在第一次出现的时候会有比较详细的安装和使用说明,但在后续的使用中则不再赘述。所以,建议读者按章节顺序阅读本书内容。

\item 每一讲的开头会列出本讲的内容提要,而末尾会有小结和练习题。引用的参考文献在全书末尾列出。

\item 以星号开头的章节是选读部分,读者可以根据兴趣阅读。跳过它们不会对理解后续章节产生影响。

\item 文中重要的内容以\textbf{黑体}标出,相信你已经习惯了。

\item 我们设计的实验大多数是演示性质的。看懂了它们不代表你已经熟悉整个库的使用。所以我们建议你在课外花一点时间,对本书经常用的几个库进行深入学习。

\item 本书的习题和选读内容可能需要你自己搜索额外材料,所以你需要学会使用搜索引擎。
\end{enumerate}

\section{致谢和声明}
在本书漫长的写作过程中,我得到了许多人的帮助,包括但不限于:

\begin{itemize}
	\item 中科院的贺一家博士为第5讲的相机模型部分提供了材料。
	\item 颜沁睿提供了第7讲的公式推导材料。
	\item 华中科大的刘毅博士为本书第6讲和第10讲提供了材料。
	\item 众多的老师、同学为本书提供了修改意见:肖锡臻、谢晓佳、张明明、耿欣、李帅杰、刘富强、袁梦、孙志明、陈昊升、王京、朱晏辰、丁文东、范帝楷、衡昱帆、高扬、李少朋、吴博、闫雪娇、张腾、郑帆、卢美奇、杨楠等等。在此向他们表示感谢。
\end{itemize}

此外,感谢我的导师张涛教授一直以来对我的支持和帮助。感谢电子工业出版社郑柳洁编辑的支持。没有他们的帮助,本书不可能以现在的面貌来到读者面前。本书的成书与出版是所有人共同努力的结晶,尽管我没法把他们都列在作者列表中,但是它的出版离不开他们的工作。\footnote{文章中的“我”主要是指本人高翔,前面说“我”不正经不包括上述其他作者。他们都是敬业乐群的好同志。}

本书写作过程中参考了大量文献和论文。其中大部分数学理论知识是前人研究的成果,并非我的原创。一小部分实验设计亦来自各开源代码的演示程序,不过大部分是我自己编写的。此外,也有一些图片摘自公开发表的期刊或会议论文,文中均已注明。未做说明的图像,或为原创,或来自网络,恕不一一列举。如有问题,请与我们联系,我们会在第一时间加以修正。

本书涉及知识点众多,错漏在所难免。如有疑问,欢迎通过电子邮件与我联系。

我的邮箱是:\href{mailto:gao.xiang.thu@gmail.com}{gao.xiang.thu@gmail.com}。

感谢我的爱人刘丽莲女士长期的理解和支持。这本书是献给她的。

\section*{习题(基本知识自测题)}
\begin{enumerate}
	\item 有线性方程$\bm{A} \bm{x} = \bm{b}$,若已知$\bm{A}, \bm{b}$,需要求解$\bm{x}$,该如何求解?这对$\bm{A}$和$\bm{b}$有哪些要求?提示:从$\bm{A}$的维度和秩角度来分析。
	\item 高斯分布是什么?它的一维形式是什么样子?它的高维形式是什么样子?
	\item 你知道C++中的\textbf{类}吗?你知道STL吗?你使用过它们吗?
	\item 你以前怎样书写C++程序?(你完全可以说只在Visual C++ 6.0下写过C++工程,只要你有写C++和C语言的经验就行。)
	\item 你知道C++11标准吗?其中哪些新特性你听说过或用过?有没有其他的标准?
	\item 你知道Linux吗?你有没有至少使用过一种(不算安卓),比如Ubuntu?
	\item Linux的目录结构是什么样的?你知道哪些基本命令,比如ls, cat等?
	\item 如何在Ubuntu中安装软件(不打开软件中心的情况下)?这些软件被安装在什么地方?如果只知道模糊的软件名称(比如想要装一个名称中含有eigen的库),应该如何安装它?
	\item[\optional] 花一个小时学习一下Vim,因为你迟早会用它。你可以在终端中输入vimtutor阅读一遍所有内容。我们不需要你非常熟练地操作它,只要能够在学习本书的过程中使用它输入代码即可。\textbf{不要在它的插件上浪费时间,不要想着把Vim用成IDE,我们只用它做文本编辑的工作。}
\end{enumerate}