\thispagestyle{empty}
\chapter*{第二版序}
自《视觉SLAM十四讲:从理论到实践》出版以来已经过去了一年多的时间。这一年内,这本书在业界引起了广泛的关注和讨论。大多数读者评价是正面的,不过也有些地方不够令人满意。比如说,这本书作为初学者向,入门则入门矣,有些应该深入的地方还讲得不够深入;再比如说,书的数学符号不够统一,有些地方让初学者产生了误解。还有的地方,读者认为叙述不够有趣,我没有完全放开,这都是事实,所以我准备在第二版中更加放飞自我一些。

实际上,第一版书是在2016年中期开始写的,所有文字、图片和代码都需要从头准备,所以我的精力也比较分散。2018年,我在慕尼黑工大给学生讲SLAM课程,期间又积累了一些材料。所以第二版书从内容上也会更丰富,更合理一些。第二版在第一版的基础上做的改动有:
\begin{enumerate}
\item 更多的实例。我增加了一些实验代码来介绍算法的原理。在第一版中,多数实践代码调用了各种库中的内置函数,我现在认为更加深入地介绍底层计算会更好一些。所以第二版的许多代码,在调用库函数之外,还提供了底层的实现。
\item 更深入的内容,特别是第七章至第十二章,同时也删除了一些泛泛而谈的边角料。对第一版大部分数学公式进行了审视,重写了那些容易引起误解的地方。
\item 更完善的项目。我将第一版的第九章移至了第十三章。于是,我们可以在介绍了所有必要知识之后,向大家展现一个完整的SLAM系统如何工作了。
\item 更通俗、简洁、口语化的表达,我觉得这是一本好书的标准,特别是介绍一些看起来高深莫测的数学知识时。我也重新制作了部分插图,在黑白打印成书之后会看起来更清楚一些。
\item 当然,每章前的简笔画我是不会改的!
\end{enumerate}

总之,我尽量做到深入浅出,也希望第二版书能够给你更加舒服的阅读体验。

\clearpage